\documentclass{report}
\usepackage{graphicx, siunitx, amsmath, float} % Required for inserting images
\title{Phys499A Report}
\author{Thomas Belvin}
\date{October 2023}
\begin{document}
\maketitle
\section{Introduction}
The goal is to assess the viability of using wires to detect gravitational particles 
by measuring the force applied at the end of a wire due to a wave generated by an inciting
particle.

If the particle is moving sufficiently at a sufficiently fast speed, $v$, the interaction can be approximated by the 
particle moving along a straight track in relation to the wire. I then define the coordinate system
such that the wire lies on the $\mathbf{\hat z}$ axes and the distance of closest approach between the 
wire and the particle track, $b$, is centered at $z = 0$. 
\begin{align}
    r_x &= b \cos (\phi )+t v \sin (\theta ) \sin (\phi )\\
    r_y &= b \sin (\phi )-t v \sin (\theta ) \cos (\phi )\\
    r_z &= t v \cos (\theta )-z_{\text{wire}}
\end{align}
where $\mathbf{r}$ is the vector from a point on the wire, $z_{\text{wire}}$ to the particle, $\phi$ is the angle between
the $\mathbf{\hat x}$ axes and the track at $z = 0$, and $\theta$ is the angle between track and the $\mathbf{\hat z}$ axes.
% Insert example image of axes

The force the particle applies on a small piece of the wire is
\begin{equation}
    \mathbf{F} = \frac{G M \partial m}{\mathbf{r}^2} \mathbf{\hat r}
\end{equation}
where $M$ is the mass of the inciting particle, $\partial m$ is the mass of a small segment of the wire,
and $\mathbf{r}$ is the vector between the inciting particle and the segment of wire.

Since the particle is moving quick enough that it can be approximated as moving in a straight line in relation
to the wire, it is safe to assume that it will apply and impulse to the wire. Taking $\mathbf{F}$ from (4) and 
integrating from negative infinity to infinity with time gives
\begin{align}
    I_x &= \frac{2 G M \partial m (b \cos (\phi )+z_{\text{wire}} \sin (\theta ) \cos (\theta ) \sin (\phi ))}{v \left(b^2-z_{\text{wire}}^2 \cos ^2(\theta )+z_{\text{wire}}^2\right)}\\
    I_y &= \frac{2 G M \partial m (b \sin (\phi )-z_{\text{wire}} \sin (\theta ) \cos (\theta ) \cos (\phi ))}{v \left(b^2-z_{\text{wire}}^2 \cos ^2(\theta )+z_{\text{wire}}^2\right)}\\
    I_z &= -\frac{2 G M \partial m \text{ } z_{\text{wire}} \sin ^2(\theta )}{v \left(b^2-z_{\text{wire}}^2 \cos ^2(\theta )+z_{\text{wire}}^2\right)}
\end{align}
Dividing these impulses by $\partial m$ gives an initial velocity to each point on the wire which can be used to 
solve the wave equation and show the time evolution of the wire after the impulse provided by the inciting particle.
I also used pen and paper analysis to create a travelling wave pulse $\dot{\mathbf{\psi}}$ which matches the solutions provided
by scipy's solve\_ivp method. It is easy to see that  $\frac{I_i}{\partial m} = \dot{\psi_i} (z, 0)$
\begin{align}
    \dot{\psi} (z, t) &= {\dot{\psi} (z, t)}_L + {\dot{\psi} (z, t)}_R\\
    \nonumber\\
    \nonumber {\dot{\psi_x} (z, t)}_L &= \frac{G M}{v} \frac{b \cos (\phi )+ (z - w t) \sin (\theta ) \cos (\theta ) \sin (\phi )}{b^2-{(z - w t)}^2 \sin ^2(\theta )}\\
    \nonumber {\dot{\psi_x} (z, t)}_R &= \frac{G M}{v} \frac{b \cos (\phi )+ (z + w t) \sin (\theta ) \cos (\theta ) \sin (\phi )}{b^2-{(z + w t)}^2 \sin ^2(\theta )}\\
    \nonumber\\
    \nonumber {\dot{\psi_y} (z, t)}_L &= \frac{G M}{v} \frac{(b \sin (\phi )-{(z + w t)} \sin (\theta ) \cos (\theta ) \cos (\phi ))}{\left(b^2-{(z + w t)}^2 \sin ^2(\theta )\right)}\\
    \nonumber {\dot{\psi_y} (z, t)}_R &= \frac{G M}{v} \frac{(b \sin (\phi )-{(z - w t)} \sin (\theta ) \cos (\theta ) \cos (\phi ))}{\left(b^2-{(z - w t)}^2 \sin ^2(\theta )\right)}
    \nonumber\\
    \nonumber {\dot{\psi_z} (z, t)}_L &= \frac{G M}{v} \frac{{(z + w t)} \sin ^2(\theta )}{\left(b^2-{(z + w t)}^2 \sin ^2(\theta )\right)}\\
    \nonumber {\dot{\psi_z} (z, t)}_R &= \frac{G M}{v} \frac{{(z - w t)} \sin ^2(\theta )}{\left(b^2-{(z - w t)}^2 \sin ^2(\theta )\right)}
\end{align}
where w is the wave speed of the wire. The $\mathbf{\hat x}$ and $\mathbf{\hat y}$ waves are transverse/shear waves and the $\mathbf{\hat z}$ is a longitudinal/pressure wave.
Taking the time derivatives of $\dot{\psi}$ gives the accelerations of the wavepulses;
\begin{align}
    \ddot{\psi} (z, t) =& {\ddot{\psi} (z, t)}_L + {\ddot{\psi} (z, t)}_R\\
    \nonumber {\ddot{\psi_x} (z, t)}_R =& -\frac{G M w \sin (\theta ) \cos (\theta ) \sin (\phi )}{v \left(b^2+\sin ^2(\theta ) {(z - w t)}^2\right) }\\
    & +\frac{2 G M w \sin ^2(\theta ) (z - w t) (b \cos (\phi )+\sin (\theta ) \cos (\theta ) \sin (\phi ) (z - w t))}{v {\left(b^2+\sin ^2(\theta ) {(z - w t)}^2\right)}^2}\\
    \nonumber {\ddot{\psi_y} (z, t)}_R =& \frac{G M w \sin (\theta ) \cos (\theta ) \cos (\phi )}{v \left(b^2+\sin ^2(\theta ) {(z - w t)}^2\right)}\\
    & +\frac{2 G M w \sin ^2(\theta ) (z - w t) (b \sin (\phi )-\sin (\theta ) \cos (\theta ) \cos (\phi ) (z - w t))}{v {\left(b^2+\sin ^2(\theta ) {(z - w t)}^2\right)}^2} \\
    \nonumber {\ddot{\psi_y} (z, t)}_R =& \frac{G M w \sin ^2(\theta )}{v \left(b^2+\sin ^2(\theta ) {(z + w t)}^2\right)} - \frac{2 G M w \sin ^4(\theta ) {(z - w t)}^2}{v {\left(b^2+\sin ^2(\theta ) {(z + w t)}^2\right)}^2}\\
\end{align} 
The left wave pulse accelerations have the same structure as the right except with the signs of each fraction and wavespeed flipped.
It should also be noted that when plugging values into a computer I factored out the $\frac{G M}{v}$ term to preven IEEE precision errors.
% Convert to ivp in wave equation, show agreement between theory and numerical analysis


\section{Example}
The example that best shows each individual waveform is with $\phi = 90^\circ$ and $\theta = 45^\circ$.

\begin{figure}[H]
    \includegraphics[width=\linewidth]{DifferentBsWithForceApprox.png}
\end{figure}

This figure shows that you can fit $b \text{, } \phi \text{, and} \theta$ to the shape of the wave and then find $\frac{G M}{v}$ from the amplitude of force.
The maximum force (without the GM/v component) caused by each wave has been approximated by taking the FWHM * linear mass density * wave speed. This combined with 
the magnitude of the $\frac{G M}{v}$ component can give us an estimation for how sensitive our force sensors would have to be.

Based on **Reference to writing on force sensors** most force sensors work best within a range of milli-Newtons, with more sensitive ones being able to detect micro-Newtons.
The most sensitive force sensor I found was able to detect about 10 zepto-Newtons ** Reference **. 

Based off the figure, the maximum detected force is around the order of $10^2$ to $10^4$ $\times \frac{G M}{v}$ Newtons. This gives somethings around $10^{-9}$ to $10^{-7}$ $\times \frac{M}{v}$. 
The inciting particle is likely to be travelling somewhere between $220 \text{ } \frac{km}{s}$ and $3e5 \text{ } \frac{km}{s}$ leaving us with a force of approximately $10^{-17}$ to $10^{-12}$ $\times M$ Newtons.
In order for this to be detectable with one of the detectors mentioned in ** Reference article going over force sensors ** we would need a particle mass somewhere on the order of $10^{14}$ to $10^{6}$ kilograms. 
With a specialized sensor such as the one from ** other reference ** we could allow particle mass to be as low as $10^{-6}$ to $10^{-3}$ Newtons. 

By maximizing the wire Tension and choosing a wire with a large Elastic Modulus and mass density we may be able to increase the range of detectable particle masses, as the wave speed and linear mass density are proportional to 
the magnitude of the force.
\end{document}