\documentclass{report}
\usepackage{graphicx, siunitx, amsmath} % Required for inserting images
\title{Phys499A Report}
\author{Thomas Belvin}
\date{October 2023}
\begin{document}
\maketitle
\section{Introduction}
The goal is to assess the viability of using wires to detect gravitational particles 
by measuring the force applied at the end of a wire due to a wave generated by an inciting
particle.

If the particle is moving sufficiently at a sufficiently fast speed, $v$, the interaction can be approximated by the 
particle moving along a straight track in relation to the wire. I then define the coordinate system
such that the wire lies on the $\mathbf{\hat z}$ axes and the distance of closest approach between the 
wire and the particle track, $b$, is centered at $z = 0$. 
\begin{align}
    r_x &= b \cos (\phi )+t v \sin (\theta ) \sin (\phi )\\
    r_y &= b \sin (\phi )-t v \sin (\theta ) \cos (\phi )\\
    r_z &= t v \cos (\theta )-z_{\text{wire}}
\end{align}
where $\mathbf{r}$ is the vector from a point on the wire, $z_{\text{wire}}$ to the particle, $\phi$ is the angle between
the $\mathbf{\hat x}$ axes and the track at $z = 0$, and $\theta$ is the angle between track and the $\mathbf{\hat z}$ axes.
% Insert example image of axes

The force the particle applies on a small piece of the wire is
\begin{equation}
    \mathbf{F} = \frac{G M \partial m}{\mathbf{r}^2} \mathbf{\hat r}
\end{equation}
where $M$ is the mass of the inciting particle, $\partial m$ is the mass of a small segment of the wire,
and $\mathbf{r}$ is the vector between the inciting particle and the segment of wire.

Since the particle is moving quick enough that it can be approximated as moving in a straight line in relation
to the wire, it is safe to assume that it will apply and impulse to the wire. Taking $\mathbf{F}$ from (4) and 
integrating from negative infinity to infinity with time gives
\begin{align}
    I_x &= \frac{2 G M \partial m (b \cos (\phi )+z_{\text{wire}} \sin (\theta ) \cos (\theta ) \sin (\phi ))}{v \left(b^2-z_{\text{wire}}^2 \cos ^2(\theta )+z_{\text{wire}}^2\right)}\\
    I_y &= \frac{2 G M \partial m (b \sin (\phi )-z_{\text{wire}} \sin (\theta ) \cos (\theta ) \cos (\phi ))}{v \left(b^2-z_{\text{wire}}^2 \cos ^2(\theta )+z_{\text{wire}}^2\right)}\\
    I_z &= -\frac{2 G M \partial m \text{ } z_{\text{wire}} \sin ^2(\theta )}{v \left(b^2-z_{\text{wire}}^2 \cos ^2(\theta )+z_{\text{wire}}^2\right)}
\end{align}
% Convert to ivp in wave equation, show agreement between theory and numerical analysis
\section{Simple Example}
\end{document}